%%%%%%%%%%%%%%%%%%%%%%%%%%%%%%%%%%%%%%%%% -*- coding: utf-8; mode: latex -*- %%
%
%    NOTA IMPORTANTE: este ficheiro destina-se a ser formatado para
%    visualização interactiva, utilizando um programa do tipo ``netscape''
%    e foi concebido tendo em vista a seu processamento para obtenção de
%    uma especificação HTML.
%
%    ESTE ERA O COMENTÁRIO ANTES DE REUTILIZAÇÃO DO FORMATO EM LaTeX2e...
%
%%%%%%%%%%%%%%%%%%%%%%%%%%%%%%%%%%%%%%%%%%%%%%%%%%%%%%%%%%%%%%%%%%%%%%%%%%%%%%%

% $Id: 0000-preliminar.tex,v 1.2 2009/06/14 20:11:06 david Exp $
% $Log: 0000-preliminar.tex,v $
% Revision 1.2  2009/06/14 20:11:06  david
% PDF version. Color index.
%
% Revision 1.1  2007/11/23 09:52:39  david
% *** empty log message ***
%
%

  %%%%%%%%%%%%%%%%%%%%%%%%%%%%%%%%%%%%%%%%%%%%%%%%%%%%%%%%%%%%%%%%%%%%%%%%%%%%%
  %
%%%%%                  TÍTULO E DATA OFICIAL DA TESE
 %%%
  %

\def\date{September 2014}
\def\titulo{Snapshotting in Hadoop Distributed File System for Hadoop Open Platform as Service}

% hypernavigation in PDF docs
\hypersetup{colorlinks,
   debug=false,
   linkcolor=blue,  %%% cor do tableofcontents, \ref, \footnote, etc
   citecolor=red,  %%% cor do \cite
   urlcolor=blue,   %%% cor do \url e \href
   bookmarksopen=true,
   pdftitle={\titulo},
   pdfauthor={Author's Name},
   pdfsubject={Labore et Dolore},
   pdfkeywords={Labore, Dolore}
}

  %%%%%%%%%%%%%%%%%%%%%%%%%%%%%%%%%%%%%%%%%%%%%%%%%%%%%%%%%%%%%%%%%%%%%%%%%%%%%
  %
%%%%%                          CAPA DA TESE
 %%%
  %

\thispagestyle{empty}

\begin{singlespace}
\vbox to\textheight{%
%--------------------------------------------------
\vskip-1.3in%---------- LOGO E NOME IST/UTL -------
%--------------------------------------------------
\hskip-17mm\vbox to50mm{
\vfil
\begin{tabular}{l}
\includegraphics[width=9cm]{figs/preliminar/Logo_IST_web.pdf}
\end{tabular}
\vfil
\vfil
}%
%--------------------------------------------------
\vskip18mm%---------- FIGURAS DA CAPA -------------
%--------------------------------------------------
\vbox to25mm{\LARGE\sl
\vfil
%\centerline{\psfig{file=figs/preliminar/tarantula.eps,height=25mm}}
\vfil
}%
%--------------------------------------------------
\vskip6mm%---------- TÍTULO -----------------------
%--------------------------------------------------
\vbox to25mm{\LARGE\bf
\vfil
\begin{center}
\titulo
\end{center}
\vfil
}%
%--------------------------------------------------
\vskip10mm%---------- NOME E GRAU ACTUAL -----------
%--------------------------------------------------
\vbox to25mm{\large
\vfil
\begin{center}
{\Large\bf Pushparaj Motamari}\\   % author's name
\end{center}
\vfil
}%
%--------------------------------------------------
\vskip8mm%---------- GRAU A OBTER -----------------
%--------------------------------------------------
\vbox to8mm{\large
\vfil
\centerline{
Dissertation for the degree of Master of}
%\vskip6mm
\centerline{Science in Distributed Computing}
\vfil
}%
%--------------------------------------------------
\vskip10mm%---------- ORIENTADOR -------------------
%--------------------------------------------------
%\vbox to8mm{\large
%\vfil
%\begin{center}
%\begin{tabular}{p{0.2\textwidth}l}
%\end{tabular}
%\end{center}
%\vfil
%}%
%%--------------------------------------------------
%\vfil
% %--------------------------------------------------
% \vskip5mm%---------- JÚRI -------------------------
% %--------------------------------------------------
\vbox to7mm{\Large\bf
\vfil
\begin{center}
{\Large\bf Jury}\\
\end{center}
\vfil
}%

\vbox to28mm{\large
\vfil
\begin{center}
\begin{tabular}{p{0.2\textwidth}l}
Presidente: & Prof. José Carlos Alves Pereira Monteiro\\
Orientador: & Prof. Luís Manuel Antunes Veiga\\
Vogal: & Prof. João Manuel dos Santos Lourenço\\
        
\end{tabular}
\end{center}
\vfil
}%
%--------------------------------------------------
\vskip28mm%---------- DATA -------------------------
%--------------------------------------------------
\vbox to4mm{\Large\bf
\vfil
\begin{center}
\date
\end{center}
\vfil
}%
%--------------------------------------------------
}%vbox
\end{singlespace}
\newpage

  %%%%%%%%%%%%%%%%%%%%%%%%%%%%%%%%%%%%%%%%%%%%%%%%%%%%%%%%%%%%%%%%%%%%%%%%%%%%%
  %
%%%%%                             AGRADECIMENTOS
 %%%
  %
\chapter*{European Master in Distributed Computing (EMDC)  }

This thesis is a part of the curricula of the European Master in Distributed Computing, a cooperation between KTH Royal Institute of Technology in Sweden,Instituto Superior Tecnico (IST)  in Portugal and Universitat Politecnica de Catalunya (UPC)  in Spain. This double degree master program is supported by Education,Audiovisual and Culture Executive Agency (EACEA)  of the European Union.\\\\
My Study track during the master studies of the two years is as follows:\\\\
First Year: Instituto Superiot Tecnico,Universidade de Lisboa\\
Third Semester:KTH Royal Institute of Technology\\
Fourth Semester: SICS (Swedish Institute of Computer Science)  and Instituto Superior Tecnico, Universidade de Lisboa


\chapter*{Acknowledgements}
%\chapter*{Acknowledgements}
\thispagestyle{empty}

% AGRADECER!
I would like to express my deepest gratitude to my supervisors Dr. Luis Veiga and Dr. Jim Dowling, who were a great source of inspiration and motivation. Also, I
would like to thank my advisers Mahmoud Ismail and Salman Niazi for countless
hours of support.\\
\hspace{4 em}I would like to thank  Instituto Superior Technico for providing me opportunity to learn and excel in distributed computing.Finally, I would like to thank Swedish Institute of Computer Science (SICS)  for providing me with a nice working environment, all necessary compute
resources, and great colleagues who always willing to exchange ideas.


\vfill
\begin{flushright}
  \begin{minipage}{8cm}
    \begin{center}
      Lisboa, \today

      Pushparaj Motamari
    \end{center}
  \end{minipage}
\end{flushright}



  %%%%%%%%%%%%%%%%%%%%%%%%%%%%%%%%%%%%%%%%%%%%%%%%%%%%%%%%%%%%%%%%%%%%%%%%%%%%%
  %
%%%%%                            DEDICATÓRIAS
 %%%
  %


% DEDICAR!




  %%%%%%%%%%%%%%%%%%%%%%%%%%%%%%%%%%%%%%%%%%%%%%%%%%%%%%%%%%%%%%%%%%%%%%%%%%%%%
  %
%%%%%                                RESUMO
 %%%
  %



  %%%%%%%%%%%%%%%%%%%%%%%%%%%%%%%%%%%%%%%%%%%%%%%%%%%%%%%%%%%%%%%%%%%%%%%%%%%%%
  %
%%%%%                            ABSTRACT
 %%%
  %

\chapter*{Abstract}
\thispagestyle{empty}
The amount of data stored in modern data centres is growing rapidly nowadays.
Large-scale distributed file systems, that maintain the massive data sets in data
centres, are designed to work with commodity hardware. Due to the quality and
quantity of the hardware components in such systems, failures are considered
normal events and, as such, distributed file systems are designed to be highly
fault-tolerant. \\\\
\hspace{4em}     A concrete implementation of such a file system is the Hadoop Distributed File
System  (HDFS) . Snapshot means capturing the state of the storage system at an exact point in time and is used to provide full recovery of data when lost. Operational as well as analytical applications manipulate the data in the distributed file system on behalf of the user or the administrator. Application-level errors or even inadvertent user errors can mistakenly delete data or modify data in an unexpected way. In this case, snapshots can be used to recover to a known, well-defined state. Snapshots can be used in Model Training, Managing Real-time Data Analysis and also to produce backups on the fly (Hot Backups) . We designed and implemented nested snapshots which enables multiple snapshots on any directory. We designed and implemented root level single snapshot by which roll-back during software upgrade can be made. We evaluate our designs and algorithms, and we show that time to take snapshot is constant and roll-back time is proportional to the changes since the snapshot was taken.

\chapter*{Resumo}
\thispagestyle{empty}
A quantidade de dados armazenados em centros de dados modernos cresce rapidamente hoje em dia. Sistemas distribuídos de larga escala, que mantêm grandes conjuntos de dados em centros de dados, são projetados para  trabalhar com hardware comum. Devido à qualidade e quantidade dos componentes de hardware nesses sistemas, as faltas são consideradas eventos normais e, como tal, os sistemas distribuídos de ficheiros são projetados para ser altamente tolerante a faltas.

Uma realização concreta de um tal sistema é o Hadoop Distributed File System  (HDFS) . Um snapshot consiste em capturar o estado do sistema de armazenamento num ponto exacto no tempo e pode ser utilizado para permitir a recuperação total dos dados quando ocorre uma falha. As aplicações manipulam os dados no sistema de ficheiros distribuído em nome de utilizadores ou administradores. Erros ao nível aplicacional ou até memso dos utilizadores podem remover informação por engano ou modificar dados de uma forma inesperada. Neste caso, os snapshots podem ser utilizados posteriormente para recuperar o sistema com o estado de um ponto anterior. Estes podem ser usados no treino de modelos, em anaálise de dados em tempo real, e também para backups rápidos  (Hot Backups) .

Desenhámos e realizámos um mecanismso de snaphsots aninhados que permite vários snapshots em qualquer pasta. O snapshot de nível raiz permite o roll-back durante a actualização de software. Avaliámos os nossos mecanismos e algoritmos, demonstrando tempo para tirar um snapshot é constante e o tempo de roll-back é proporcional à quantidade de modifcações desde o snapshot.



\newpage

  %%%%%%%%%%%%%%%%%%%%%%%%%%%%%%%%%%%%%%%%%%%%%%%%%%%%%%%%%%%%%%%%%%%%%%%%%%%%%
  %
%%%%%                 FICHA BIBLIOGRAFICA -- PALAVRAS CHAVE
 %%%
  %



\section*{Keywords}

{\large % EM INGLÊS

\noindent Hadoop

\noindent HDFS

\noindent Distributed FileSystem

\noindent Snapshots

\noindent HOPS

}

\section*{Palavras Chave}

{\large % EM INGLÊS

\noindent Hadoop

\noindent HDFS

\noindent sistema de ficheiros distribuído

\noindent Snapshots

\noindent HOPS

}


\vfill
%LATEX2HTML}

\cleardoublepage


  %%%%%%%%%%%%%%%%%%%%%%%%%%%%%%%%%%%%%%%%%%%%%%%%%%%%%%%%%%%%%%%%%%%%%%%%%%%%%
  %
%%%%%                         MUDANÇA DE NUMERAÇÃO
 %%%
  %

\pagestyle{plain}
\pagenumbering{roman}

  %%%%%%%%%%%%%%%%%%%%%%%%%%%%%%%%%%%%%%%%%%%%%%%%%%%%%%%%%%%%%%%%%%%%%%%%%%%%%
  %
%%%%%                             INDICES
 %%%
  %

% ``Table of contents'' (índice) .

\def\contentsname{Index}
\tableofcontents
\newpage

% Lista de figuras.
\listoffigures
\newpage

% Lista de tabelas.
\listoftables


% Does it always work? I expect so...
\cleardoublepage

  %
 %%%
%%%%%                          F          I          M      
  %
  %%%%%%%%%%%%%%%%%%%%%%%%%%%%%%%%%%%%%%%%%%%%%%%%%%%%%%%%%%%%%%%%%%%%%%%%%%%%%

% Local Variables: 
% mode: latex
% TeX-master: "tese"
% End: 
