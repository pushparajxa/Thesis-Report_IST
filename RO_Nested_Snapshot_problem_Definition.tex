  %%%%%%%%%%%%%%%%%%%%%%%%%%%%%%%%%%%%%%% -*- coding: utf-8; mode: latex -*- %%
  %
%%%%%                         CHAPTER
 %%%
  %

% $Id: 1020-lorem-ipsum.tex,v 1.2 2009/06/19 15:51:46 david Exp $
% $Log: 1020-lorem-ipsum.tex,v $
% Revision 1.2  2009/06/19 15:51:46  david
% *** empty log message ***
%
% Revision 1.1  2007/11/23 09:52:39  david
% *** empty log message ***
%
%

  %%%%%%%%%%%%%%%%%%%%%%%%%%%%%%%%%%%%%%%%%%%%%%%%%%%%%%%%%%%%%%%%%%%%%%%%%%%%%
  %
%%%%%                           HEAD MATTER
 %%%
  %

\chapter{Read-Only Nested Snapshots-Problem Definition}
%\addcontentsline{lof}{chapter}{\thechapter\quad Lorem Ipsum}
%\addcontentsline{lot}{chapter}{\thechapter\quad Lorem Ipsum}
\label{ch:RONSProblem}

Users typically run experiments on the data they store in HDFS. Once an experiment overwrites or deletes existing files, it is not possible to revert to the previous state of data to run new experiments. In some cases users may like to take have different snapshots of data generated by different experiments. In some cases user may like to take snapshots and parent/child directories called nested snapshots.\\\\
Snapshots are point in time images of the file system. Snapshots should cover the following elements of the file system.
\begin{enumerate}
\item Snapshot of a subtree of the file system.
\item Snapshot of the entire file system.
\end{enumerate}

\section{UseCase Scenarios}
\begin{enumerate}
\item \textbf{Protection against user errors:} Admin sets up a process to take RO snapshots periodically in a
rolling manner so that there are always x number of RO snapshots on HDFS. If a user
accidentally deletes a file, the file can be restored from the latest RO snapshot.
\item \textbf{Backup:} Admin wants to do a backup of a dataset. Depending on the requirements, admin takes
a read-only (henceforth referred to as RO) snapshot in HDFS. This RO snapshot is then read and
data is sent across to the remote backup location.
\item \textbf{Experimental/Test setups:} A user wants to test an application against the main dataset. Normally, without doing a full copy of the dataset, this is a very risky proposition since the test setups can corrupt/overwrite production data. Admin creates a read-write (henceforth referred to as RW) snapshot of the production dataset and assigns the RW snapshot to the user to be used for experiment. Changes done to the RW snapshot will not be reflected on the production dataset.

\end{enumerate}







  %%%%%%%%%%%%%%%%%%%%%%%%%%%%%%%%%%%%%%%%%%%%%%%%%%%%%%%%%%%%%%%%%%%%%%%%%%%%%
  %
%%%%%                        FIRST SECTION
 %%%
  %

\section{Related Work}

\subsection{Apache Hadoop Version 2}
Apache Hadoop implemented snapshots in their latest version \cite{Hadoop2} , which supports nested snapshots and constant time for the creation of snapshots. Since metadata and data are separated it supports string consistency for metadata snapshot but not for data.In case of file being written, unless the datanode notifies the namenode or client notifies namenode of latest length of last block via hsync, namenode is unaware of current length of the file when snapshot was being taken.The solution also couldn't address the case of replication change of half-filled last block after taking snapshot, where it is appended after taking snapshot.\\



\subsection{Hadoop at Facebook}
Facebook has implemented a solution to Snapshots \cite{Facebook}. The solution scales linearly with the number of inodes(file or directories) in the filesystem.It uses a selective copy-on-append scheme that minimizes the number of copy-on-write operations. This optimization is made possible by taking advantage of the restricted interface exposed by HDFS, which limits the write operations to appends and truncates only.The solution has space overhead since Whenever a snap-
shot is taken, a node is created in the snapshot tree  that keeps track of all the files in the
namespace by maintaining a list of its blocks IDs along with their unique generation timestamps.
If a snapshot was taken while file is being written, after the write is finished and data node notifies namenode, it saves the location of the file from  where it started writing not exactly saving the location in file when snapshot was taken.


  %%%%%%%%%%%%%%%%%%%%%%%%%%%%%%%%%%%%%%%%%%%%%%%%%%%%%%%%%%%%%%%%%%%%%%%%%%%%%
  %
%%%%%                      SECOND SECTION
 %%%
  %



  %%%%%%%%%%%%%%%%%%%%%%%%%%%%%%%%%%%%%%%%%%%%%%%%%%%%%%%%%%%%%%%%%%%%%%%%%%%%%
  %
%%%%%                          LAST SECTION
 %%%
  %
 %%%
%%%%%                        THE END
  %
  %%%%%%%%%%%%%%%%%%%%%%%%%%%%%%%%%%%%%%%%%%%%%%%%%%%%%%%%%%%%%%%%%%%%%%%%%%%%%

%%% Local Variables: 
%%% mode: latex
%%% TeX-master: "tese"
%%% End: 
