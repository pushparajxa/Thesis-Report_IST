\documentclass[a4paper,12pt]{report}
\usepackage[T1]{fontenc}
\usepackage{textcomp}
%\usepackage{endnotes}
\usepackage[latin1]{inputenc}
\usepackage[german,french,american]{babel}
\usepackage[autostyle]{csquotes}
%\usepackage[document]{ragged2e}
\usepackage[authordate15,backend=biber,autolang=hyphen,%
bibencoding=latin1,strict]{biblatex-chicago}
% \usepackage[style=chicago-authordate,backend=biber,usecompiler=true,%
% babel=hyphen,bibencoding=auto,sorting=nyt,cmslos,autocite=inline]{biblatex}
%\renewcommand*{\rmdefault}{fgn}% The font (gentium) used for pdf
\usepackage{ifthen}
\usepackage{setspace}
\usepackage{vmargin} \setpapersize{A4}
\setmarginsrb{1in}{20pt}{1in}{.5in}{1pt}{2pt}{0pt}{2mm}
% \renewcommand*{\biburlsetup}{%
%   \Urlmuskip=0mu plus 2mu\relax
%   \mathchardef\UrlBreakPenalty=200\relax
%   \mathchardef\UrlBigBreakPenalty=100\relax
%   \mathchardef\UrlEmergencyPenalty=9000\relax
%   \appto\UrlSpecials{%
%     \do\0{\mathchar`\0\penalty\UrlEmergencyPenalty}%
%     \do\1{\mathchar`\1\penalty\UrlEmergencyPenalty}%
%     \do\2{\mathchar`\2\penalty\UrlEmergencyPenalty}%
%     \do\3{\mathchar`\3\penalty\UrlEmergencyPenalty}%
%     \do\4{\mathchar`\4\penalty\UrlEmergencyPenalty}%
%     \do\5{\mathchar`\5\penalty\UrlEmergencyPenalty}%
%     \do\6{\mathchar`\6\penalty\UrlEmergencyPenalty}%
%     \do\7{\mathchar`\7\penalty\UrlEmergencyPenalty}%
%     \do\8{\mathchar`\8\penalty\UrlEmergencyPenalty}%
%     \do\9{\mathchar`\9\penalty\UrlEmergencyPenalty}}%
%   \def\UrlBreaks{%
%     \do\.\do\@\do\/\do\\\do\!\do\_\do\|\do\;\do\>\do\]\do\)\do\}%
%     \do\,\do\?\do\'\do\+\do\=\do\#\do\$\do\&\do\*\do\^\do\"}%
%   \def\UrlBigBreaks{\do\:\do\-}}
\usepackage{url}
\urlstyle{rm}
\appto\bibsetup{\sloppy}
\hyphenation{evans-ton clem-ens}
\setlength{\dimen\footins}{9.5in}
\setlength{\parindent}{0pt}
\setlength{\parskip}{5pt}
\setcounter{biburlnumpenalty}{9000}
\setcounter{biburlucpenalty}{9000}
\setcounter{biburllcpenalty}{9000}
\newcommand{\cmd}[1]{\texttt{\textbackslash #1}}
\usepackage[colorlinks,urlcolor=blue,citecolor=black,
plainpages=false,breaklinks=true]{hyperref}
\bibliography{dates-test}
%%\onehalfspacing
\begin{document}

\section*{The Chicago Author-Date Specification, 15th Edition}
\label{sec:spec}

\subsection*{Important Note}
\label{bibernote}

Starting with \textsf{biblatex} version 1.5, in order to adhere to the
author-date specification you will need to use \textsf{Biber} to
process your .bib files, as \textsc{Bib}\TeX\ (and its more recent
variants) will no longer provide all the features you need.  I highly
recommend, therefore, that you upgrade either to \textsf{Biber} 0.9.9
and to \textsf{biblatex} 1.7, which are designed to work together, or
to \textsf{Biber} 1.8 and \textsf{biblatex} 2.8a, which latter two are
the newest releases and are likewise designed to work as a matched
pair.  The advice that follows in this document assumes that you are
using \textsf{Biber}; if you wish to continue using \textsc{Bib}\TeX\
then you need \textsf{biblatex} version 1.4c and
\textsf{biblatex-chicago} 0.9.7a.

\subsection*{Usage}
\label{usage}

As a general rule, you'll probably want to use the \cmd{autocite}
command for most citations.  For most sources, the result will be
exactly as you expect it to be.  A few examples:
\autocite{adorno:benj}; \autocite{ashbrook:brain};
\autocite{babb:peru}; \autocite{barcott:review:15}.  Any page
references should also appear as you expect: \autocite[338]{batson};
\autocite[79]{beattie:crime}; \autocite[36]{boxer:china}.

\subsection*{Repeated citations}
\label{sec:ibidem}

Repeated citations are somewhat complicated.  The Chicago author-date
style doesn't use \enquote{\emph{Ibid},} but in general a repeated
citation on the same page will print only the page reference:
\autocite{browning:aurora}; \autocite[45]{browning:aurora}.
Technically, this should only occur when a source is cited
\enquote{more than once in one paragraph}
\autocite[16.114]{chicago:manual:15}, so you can use the
\cmd{citereset} command from \textsf{biblatex} to achieve the greatest
compliance, as the package only offers automatic resetting on part,
chapter, section, and subsection boundaries, while
\textsf{biblatex-chicago} automatically resets the tracker at page
breaks:

\citereset\cmd{citereset}\ \autocite[16.115]{chicago:manual:15}.  If
you are going to repeat a source, make sure that the cite command
provides a postnote --- if you don't need to cite a specific page,
then it's better only to use one citation rather than two, as
otherwise, in the current state of the code, you'll get empty
parentheses, like so: \autocite{chicago:manual:15}.

\subsection*{Other citation commands}
\label{sec:other}

The other citation commands from \textsf{biblatex} also work fine:

\cmd{textcite}: \textcite{conley:fifthgrade}; \cmd{autocite*}:
\autocite*{connell:chronic}; \cmd{cite}: \cite{conway:evolution};
\cmd{cite*}: \cite*{davenport:attention}; \cmd{foot\-note} with
\cmd{autocite};\footnote{\autocite{donne:var:15}.}\ \cmd{footcite}
(=\cmd{cite} inside a \cmd{footnote}).  \footcite{dunn:revolutions}

Multicites should work as you expect, too:

\cmd{autocites}: \autocites{dyna:browser}{eliot:pound};
\cmd{autocites} by the same author:
\autocites{pirumova}{pirumova:russian}; \cmd{autocites} by the same
author with postnotes: \autocites{pirumova}[14]{pirumova:russian};
\cmd{textcites} by the same author with postnotes:
\textcites[37]{pirumova}{pirumova:russian}.

\subsection*{Shorthands}
\label{sec:shorthands}

Chicago's author-date style only seems to recommend the use of
shorthands as abbreviations for long authors' names, particularly
institutional names \autocite[17.47]{chicago:manual:15}.  By default, I
have followed this recommendation: \cmd{autocites}:
\autocites{bsi:abbreviation:15}{iso:electrodoc:15}; \cmd{textcites}:
\textcites{bsi:abbreviation:15}{iso:electrodoc:15}.  It suggests
placing the expansion of the abbreviation into an alphabetized
cross-reference inside the reference list itself, a procedure that I
have now implemented for \textsf{biblatex-chicago}.  To use this, you
need a \textsf{CustomC} entry containing the expansion, and either a
\textsf{userc} field in the parent entry or a \cmd{nocite} command to
make sure it is printed in the reference list.  If you use a
\cmd{printshorthands} command, the list of shorthands will still be
printed, so you now have a variety of options available for presenting
the expansions depending on your specific requirements.  Please note,
also, that you can get back something approaching the
\enquote{standard} behavior of shorthands if you give the
\texttt{cmslos=false} option to \textsf{biblatex-chicago} in your
document preamble.

\subsection*{Mildly problematic entries}
\label{sec:problematic}

In most entries, the absence of an author can be supplied by, e.g., an
editor or a translator: \autocite{chaucer:alt};
\autocite{silver:gawain}.  Sometimes an anonymous work's author is
known or can be guessed: \autocite{horsley:prosodies};
\autocite{cook:sotweed}.  Alternatively, in some cases the use of
\enquote{\texttt{Anon.}}\ as author may be the simplest option:
\autocite{anon:stanze:15}; \autocite{virginia:plantation:15}.

With \textsf{Biber}, an absent \textsf{date} will automatically
provoke it into searching for other sorts of dates in the entry, in
the order \textsf{year, eventyear, origyear, urlyear}: e.g.,
\autocite{evanston:library}, which only has a \textsf{urlyear}.  (You
can eliminate some of these dates from the running, or change the
search order, using the \cmd{DeclareLabelyear} command in your
preamble, but please be aware that I have hard-coded this order into
the author-date style in order to cope with some tricky corners of the
specification.  If you reorder these dates, and your references enter
these tricky corners, the results might be surprising.  Cf.\
section~4.5.2 in \textsf{biblatex.pdf} and section~5.2, s.v.\
\enquote{date} in \textsf{biblatex-chicago15.pdf} for the details.)
In most entry types, the absence of all four possible dates will
automatically produce \enquote{\texttt{n.d.}\hspace{-2pt}} instead:
\autocite{bernstein:shostakovich}.  You can also give it yourself in
the form \cmd{bibstring\{nodate\}}: \autocite{ross:thesis}.  A date
that can be guessed should appear within square brackets:
\autocite{clark:mesopot}.  Forthcoming works are straightforward,
assuming you remember to use the \cmd{autocap} macro so that the word
appears correctly in both citations and the list of references:
\autocite{author:forthcoming}; \autocite{contrib:contrib}.

The \emph{Manual} outlines a series of options for entries with more
than one date \autocite[17.124--27]{chicago:manual:15}.  All of these
possibilities are available in \textsf{biblatex-chicago} using the
\texttt{cmsdate} entry option: \texttt{cmsdate=off} (the default):
\autocite{maitland:equity}; \texttt{cmsdate=on}:
\autocite{james:ambassadors}; \texttt{cmsdate=old}:
\autocite{emerson:nature}; \texttt{cmsdate=new}:
\autocite{maitland:canon}.  These options, in combination with others
available in your .bib files, can cover a wide range of difficult
cases.  Please see the next section below, the documentation in
\textsf{biblatex-chicago15.pdf}, and also the following entries in
\textsf{dates-test.bib}:
\autocites{schweitzer:bach}{white:russ}{white:ross:memo}.

\subsection*{Corners of the specification}
\label{sec:corners}

In some cases, the \emph{Manual} isn't altogether clear about how to
present entries in the author-date style.  I'm pretty certain about
most of what follows, but if you interpret the specification
differently please let me know.

\subsubsection*{InReference entries}
\label{sec:inref}

These present several peculiarities: the title of the work should
always take the place of any author, no
\enquote{\texttt{n.d.}\hspace{-2pt}} will be automatically provided,
and any postnote field will be enclosed in quotation marks preceded by
\enquote{\texttt{s.v.}\hspace{-2pt}} for \enquote{\emph{sub verbo}.}
This allows you to refer to alphabetized articles in well-known
reference works: \autocite[Hume, David]{ency:britannica};
\autocite[Sibelius, Jean]{grove:sibelius};
\autocite[BibTeX]{wikiped:bibtex}.

\subsubsection*{Author-less Article (and Manual) entries}
\label{sec:authless:art}

In article entries with the \texttt{magazine} entrysubtype, the
absence of an author automatically places the title of the periodical
in citations and at the head of the entry in the list of references:
\autocite{gourmet:052006}.  You can also use the new
\texttt{cmsdate=full} switch in the \textsf{options} field to produce
complete date references in such entries, thereby obviating any need
to present them in the list of references at all:
\autocite{lakeforester:pushcarts}; \autocite{nyt:trevorobit}.  In
\textsf{manual} entries, the \textsf{organization} field can take the
place of a missing \textsf{author}: \autocite{dyna:browser}.  If you
wish to present an abbreviated form of the journal (or organization)
name only in citations, then the \textsf{shortauthor} field --- or in
other cases the \textsf{shorthand} field --- is the place for it,
making sure to include formatting: \autocite{unsigned:ranke};
\autocite{bsi:abbreviation:15}.

\subsubsection*{Misc entries with an entrysubtype}
\label{sec:misc}

When citing individual letter-like pieces from an unpublished archive
where only an \textsf{origdate} is present, you no longer need to set
the \texttt{cmsdate} option in your .bib entry, as \textsf{Biber} and
\textsf{biblatex-chicago} now handle this automatically:
\autocite{creel:house}.  Non-letters, e.g., interviews, use the
\textsf{date} field, so you don't need \texttt{cmsdate} there, either:
\autocite{spock:interview}.  For undated pieces you can put
\cmd{bibstring\{nodate\}} in the \textsf{year} field:
\autocite{dinkel:agassiz}.  For citing whole collections, see the next
section.

\subsubsection*{entrysubtype = \{classical\}}
\label{sec:classical}

This option's name derives from its use for citing texts from
classical antiquity, though in the author-date style especially it can
be put to use in several other contexts.  In a nutshell, any entry
with such an \textsf{entrysubtype} will be treated, in citations only,
not as author-date but as author-title.  (Entries in the list of
references, e.g., a particular edition of Aristotle, will still appear
in standard author-date format.)  A \cmd{cite*} or \cmd{autocite*}
command will, in such a case, produce the title rather than the year.
Some examples should make this clearer:

Classical works: without abbreviation:
\autocite{aristotle:metaphy:trans}; with abbreviation:
\autocite{aristotle:metaphy:gr}; \autocite{plato:republic:gr}; using
standard pagination: \autocite*[3.2.996b5--8]{aristotle:metaphy:gr};
\autocite*[420e]{plato:republic:gr}; work cited by page of a modern
edition, i.e., without \textsf{entrysubtype}:
\autocite[198]{euripides:orestes}.

Sacred works, e.g., the Bible and the Qur'an:
\autocite[25:19--36:43]{genesis}.

An unpublished archive, from which more than one work has been cited:
\autocite[file 12]{house:papers}.  (Both this and the previous example
use a Misc entry with \texttt{classical} \textsf{entrysubtype}.)

\subsubsection*{Comments inside citations}
\label{sec:comments}

If you wish to include a comment inside the parentheses of a citation,
it will need to be separated by a semicolon
\autocite[16.111]{chicago:manual:15}.  If you have a
\textsf{postnote}, then you can manually provide the punctuation and
comment in that field, e.g., \autocite[4; the unrevised
trans.]{stendhal:parma}.  Without a \textsf{postnote}, you'll need a
separate Misc or CustomC entry containing just the text of the comment
in the \textsf{title} field, \textsf{entrysubtype} \texttt{classical},
and \textsf{options} \texttt{skipbib}.  An \cmd{autocites} command
calling both the main text and the comment will do the trick, e.g.,
\autocites{chicago:manual:15}{chicago:comment:15}.

\subsubsection*{Multiple authors}
\label{sec:multiple}

The default settings in \textsf{biblatex-chicago} are
\texttt{maxnames=3,minnames=1} in citations and
\texttt{max\-bibnames=10,minbibnames=7} in the list of references
(these latter parameters set in \textsf{biblatex-chicago.sty}).  In
practice, this means that an entry like hlatky:hrt, with 5 authors,
will present all of them in the list of references but will truncate
to one in citations, like so: \autocite{hlatky:hrt}.  For the vast
majority of circumstances, these settings are exactly right for the
Chicago author-date specification.  However, if \enquote{a reference
  list includes another work \emph{of the same date} that would also
  be abbreviated as [\enquote{Hlatky et al.}] but whose coauthors are
  different persons or listed in a different order, the text citations
  must distinguish between them} \autocite[16.118]{chicago:manual:15}.
The new (\textsf{Biber}-only) \textsf{biblatex} option
\texttt{uniquelist}, set for you in \textsf{biblatex-chicago.sty},
will automatically handle many of these situations for you, but it is
as well to understand that it does so by temporarily suspending the
limits, listed above, on how many names to print in a citation.
Without \texttt{uniquelist}, \textsf{biblatex} would present such a
work as, e.g., (Hlatky et al. 2002b), while hlatky:hrt would be
(Hlatky et al. 2002a).  This does distinguish between them, but
inaccurately, as it suggests that the two different author lists are
exactly the same.  With \texttt{uniquelist}, the two citations might
look like (Hlatky, Boothroyd et al.\ 2002) and (Hlatky, Smith et al.\
2002), which is what the specification requires.

If, however, the distinguishing name occurs further down the author
list --- in fourth or fifth position in our examples --- then the
default settings would produce citations with all 4 or 5 names
printed, which can become awkwardly long.  In such a situation, you
can provide \textsf{shortauthor} fields that look like this:
\{\{Hlatky et al., \textbackslash mkbibquote\{Quality of Life,\}\}\}
and \{\{Hlatky et al., \textbackslash mkbibquote\{Depressive
Symptoms,\}\}\}, using a shortened title to distinguish the
references.  This would produce (Hlatky et al., \enquote{Quality of
  Life,} 2002) and (Hlatky et al., \enquote{Depressive Symptoms,}
2002), as the spec recommends.  There is, unfortunately, no simpler
way that I know of to deal with this situation.

\subsubsection*{Audiovisual entries}
\label{sec:audiovisual}

According to the \emph{Manual}, \enquote{the author-date system is
  inappropriate for most audiovisual materials\ldots\ Such materials
  are best mentioned in running text and grouped in the reference list
  under a subhead\ldots} \autocite[17.265]{chicago:manual:15}.  Some
entries work perfectly well: \autocite{auden:reading:15};
\autocite{hitchcock:nbynw}; \autocite{handel:messiah:15}.  Others
perhaps require further information in the entry or genuinely are
better suited to presentation in running text:
\autocite{beethoven:sonata29}; \autocite{nytrumpet:art:15}.  Published
(\textsf{Audio}) and unpublished (\textsf{Misc}) scores, for their
part, are no problem at all: \autocite{schubert:muellerin};
\autocite{verdi:corsaro}; \autocite{shapey:partita:15}.  The standard
\textsf{biblatex} tools for subdividing reference lists are all
available if you want to follow the \emph{Manual's} recommendations on
that front.

\subsection*{Further examples (mainly for testing purposes)}
\label{testing}

Article: \autocite{assocpress:gun}; \autocite{brown:bremer};
\autocite{bundy:macneil}; \autocite{chu:panda};
\autocite{Clemens:letter}; \autocite{conley:fifthgrade};
\autocite{connell:chronic}; \autocite{friedman:learning};
\autocite{garaud:gatine}; \autocite{garrett:15};
\autocite{gibbard:15}; \autocite{kern}; \autocite{kimluu:diethyl:15};
\autocite{kozinn:review}; \autocite{lewis:15};
\autocite{loften:hamlet}; \autocite{loomis:structure:15};
\autocite{morgenson:market}; \autocite{osborne:poison:15};
\autocite{ratliff:review:15}; \autocite{reaves:rosen};
\autocite{rozner:liberation}; \autocite{schneider:mittelpleistozaene};
\autocite{sewall:letter}; \autocite{stenger:privacy};
\autocite{terborgh:preservation}; \autocite{wall:radio};
\autocite{wallraff:word}; \autocite{warr:ellison};
\autocite{white:callimachus}.

Artwork: \autocite{leo:madonna}.

Audio: \autocite{greek:filmstrip:15}; \autocite{twain:audio};
\autocite{weed:flatiron}.

Book: \autocite{barrows:reading}; \autocite{churchill:letters};
\autocite{cohen:schiff}; \autocite{cotton:manufacture};
\autocite{creasey:ashe:blast}; \autocite{creasey:morton:hide};
\autocite{creasey:york:death}; \autocite{davenport:attention};
\autocite{feydeau:farces}; \autocite{furet:passing:eng};
\autocite{furet:passing:fr}; \autocite{harley:cartography};
\autocite{hopp:attalid}; \autocite{howell:marriage};
\autocite{lach:asia}; \autocite{lecarre:quest};
\autocite{levistrauss:savage}; \autocite{lynch:webstyle};
\autocite{maisonneuve:relations}; \autocite{mchugh:wake:15};
\autocite{menchu:crossing}; \autocite{meredith:letters};
\autocite{michelangelo:poems}; \autocite{mla:style};
\autocite{natrecoff:camera}; \autocite{palmatary:pottery};
\autocite{pelikan:christian}; \autocite{rodman:walk};
\autocite{schellinger:novel}; \autocite{sechzer:women};
\autocite{sereny:cries}; \autocite{soltes:georgia};
\autocite{stendhal:parma}; \autocite{suangtho:tectona};
\autocite{thompson:making}; \autocite{tillich:system};
\autocite{times:guide}; \autocite{turabian:manual};
\autocite{walker:columbia}; \autocite{wauchope:ceramics};
\autocite{weber:saugetiere}; \autocite{weresz};
\autocite{white:total}; \autocite{wright:evolution};
\autocite{wright:theory}.

BookInBook: \autocite{bernhard:boris}.

Collection: \autocite{brush:ornithology}; \autocite{kamrany:economic};
\autocite{prairie:state}; \autocite{zukowsky:chicago}.

Image: \autocite{bedford:photo}.

InBook: \autocite{ashbrook:brain}; \autocite{phibbs:diary};
\autocite{will:cohere}.

InCollection: \autocite{centinel:letters}; \autocite{ellet:galena};
\autocite{keating:dearborn}; \autocite{lippincott:chicago};
\autocite{sirosh:visualcortex}; \autocite{wiens:avian}.

InProceedings: \autocite{frede:inproc}.

InReference: \autocite[absolute]{oed:cdrom}.

Manual: \autocite{dyna:browser}.

Online: \autocite{powell:email}.

Patent: \autocite{petroff:impurity}.

Periodical: \autocite{good:wholeissue}; \autocite{whittington:water}.

Report: \autocite{herwign:office}.

SuppBook: \autocite{friedman:intro}; \autocite{polakow:afterw};
\autocite{prose:intro}.

Thesis: \autocite{murphy:silent:15}.

Unpublished: \autocite{nass:address}.

Video: \autocite{cleese:holygrail}; \autocite{episode:tv}.


% \printshorthands % No longer necessary in author-date.
%\nocite{*}
\printbibliography[title=References]

\end{document}
%%% Local Variables: 
%%% mode: latex
%%% TeX-master: t
%%% End: 
