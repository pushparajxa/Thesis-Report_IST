  %%%%%%%%%%%%%%%%%%%%%%%%%%%%%%%%%%%%%%% -*- coding: utf-8; mode: latex -*- %%
  %
%%%%%                         CHAPTER
 %%%
  %

% $Id: 1020-lorem-ipsum.tex,v 1.2 2009/06/19 15:51:46 david Exp $
% $Log: 1020-lorem-ipsum.tex,v $
% Revision 1.2  2009/06/19 15:51:46  david
% *** empty log message ***
%
% Revision 1.1  2007/11/23 09:52:39  david
% *** empty log message ***
%
%

  %%%%%%%%%%%%%%%%%%%%%%%%%%%%%%%%%%%%%%%%%%%%%%%%%%%%%%%%%%%%%%%%%%%%%%%%%%%%%
  %
%%%%%                           HEAD MATTER
 %%%
  %

\chapter{Read-Only Root Level Single Snapshot-Problem Definition}
%\addcontentsline{lof}{chapter}{\thechapter\quad Lorem Ipsum}
%\addcontentsline{lot}{chapter}{\thechapter\quad Lorem Ipsum}
\label{ch:RORLSSProblem}





  %%%%%%%%%%%%%%%%%%%%%%%%%%%%%%%%%%%%%%%%%%%%%%%%%%%%%%%%%%%%%%%%%%%%%%%%%%%%%
  %
%%%%%                        FIRST SECTION
 %%%
  %

\section{Problem-Definition}
During software upgrades the possibility of corrupting the filesystem due to software bugs or human mistakes increases. This invites a solution to  minimize potential damage to the data stored in the system during upgrades.We need to take a snapshot at root of the file system before proceeding with the upgrade.If upgrade didn't work then administrator can roll back the system to the snapshot.The snapshot can be taken at any time and can be rolled-back to at any time.
\subsection{Related Work}
Apache Hadoop distribution provides single snapshot mechanism to protected file system meta-data and storage-data from software upgrades. The snapshot mechanism lets administrators persistently save the current state of the filesystem, so that if the upgrade results in data loss or corruption it is possible to rollback the upgrade and return HDFS to the namespace and storage state as they were at the time of the snapshot.

The snapshot (only one can exist) is created at the cluster administrator's option whenever the system is started. If a snapshot is requested, the NameNode first reads the checkpoint and journal files and merges them in memory. Then it writes the new checkpoint and the empty journal to a new location, so that the old checkpoint and journal remain unchanged.

During handshake the NameNode instructs DataNodes whether to create a local snapshot. The local snapshot on the DataNode cannot be created by replicating the directories containing the data files as this would require doubling the storage capacity of every DataNode on the cluster. Instead each DataNode creates a copy of the storage directory and hard links existing block files into it. When the DataNode removes a block it removes only the hard link, and block modifications during appends use the copy-on-write technique. Thus old block replicas remain untouched in their old directories.

The cluster administrator can choose to roll back HDFS to the snapshot state when restarting the system. The NameNode recovers the checkpoint saved when the snapshot was created. DataNodes restore the previously renamed directories and initiate a background process to delete block replicas created after the snapshot was made. Having chosen to roll back, there is no provision to roll forward. The cluster administrator can recover the storage occupied by the snapshot by commanding the system to abandon the snapshot; for snapshots created during upgrade, this finalizes the software upgrade.

System evolution may lead to a change in the format of the NameNode's checkpoint and journal files, or in the data representation of block replica files on DataNodes. The layout version identifies the data representation formats, and is persistently stored in the NameNode's and the DataNodes' storage directories. During startup each node compares the layout version of the current software with the version stored in its storage directories and automatically converts data from older formats to the newer ones. The conversion requires the mandatory creation of a snapshot when the system restarts with the new software layout version.

  %%%%%%%%%%%%%%%%%%%%%%%%%%%%%%%%%%%%%%%%%%%%%%%%%%%%%%%%%%%%%%%%%%%%%%%%%%%%%
  %
%%%%%                      SECOND SECTION
 %%%
  %



  %%%%%%%%%%%%%%%%%%%%%%%%%%%%%%%%%%%%%%%%%%%%%%%%%%%%%%%%%%%%%%%%%%%%%%%%%%%%%
  %
%%%%%                          LAST SECTION
 %%%
  %


  %
 %%%
%%%%%                        THE END
  %
  %%%%%%%%%%%%%%%%%%%%%%%%%%%%%%%%%%%%%%%%%%%%%%%%%%%%%%%%%%%%%%%%%%%%%%%%%%%%%

%%% Local Variables: 
%%% mode: latex
%%% TeX-master: "tese"
%%% End: 
